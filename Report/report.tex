\documentclass[conference]{IEEEtran}
\IEEEoverridecommandlockouts
% The preceding line is only needed to identify funding in the first footnote. If that is unneeded, please comment it out.
\usepackage{cite}
\usepackage{amsmath,amssymb,amsfonts}
\usepackage{algorithmic}
\usepackage{graphicx}
\usepackage{textcomp}
\usepackage{bm}
\usepackage{hyperref}
\def\BibTeX{{\rm B\kern-.05em{\sc i\kern-.025em b}\kern-.08em
    T\kern-.1667em\lower.7ex\hbox{E}\kern-.125emX}}
    
    
\DeclareMathOperator*{\argmin}{arg\,min}
\newcommand*{\E}[1]{\ensuremath{\mathbb{E}\left[#1\right]}}
\renewcommand*{\P}[2][]{\ensuremath{\mathbb{P}#1\left(#2\right)}}
\newcommand*{\Var}[1]{\ensuremath{\text{Var}\left(#1\right)}}
\newcommand*{\Cov}[2]{\ensuremath{\text{Cov}\left({#1}, {#2}\right)}}
\newcommand*{\Corr}[1]{\ensuremath{\text{Corr}\left(#1\right)}}
\newcommand*{\sd}[1]{\ensuremath{\sigma_{#1}}}
\newcommand*{\card}[1]{\ensuremath{\lvert#1\rvert}}
\newcommand*{\warn}[1]{\textcolor{OrangeRed}{#1}}
\newcommand*{\important}[1]{\fcolorbox{BurntOrange}{BurntOrange!10!White}{\begin{varwidth}{\textwidth}
#1
\end{varwidth}}}
\renewcommand*{\d}[1]{\ensuremath{\mathrm{d}#1}}
\newcommand*{\dist}{\sim}
\newcommand*{\reals}{\ensuremath{\mathbb{R}}}
\newcommand*{\integers}{\ensuremath{\mathbb{Z}}}
\newcommand*{\naturals}{\ensuremath{\mathbb{N}}}
\newcommand*{\abs}[1]{\ensuremath{\left\lvert#1\right\rvert}}
\newcommand*{\floor}[1]{\ensuremath{\lfloor#1\rfloor}}
\newcommand*{\inclimg}[2][]{\begin{figure}[H]\centering\includegraphics[max width=0.9\textwidth,keepaspectratio]{#2}#1\end{figure}}
\newcommand*{\diff}[2]{\ensuremath{\frac{\d}{\d{#2}}{#1}}}
\newcommand*{\pdiff}[2]{\ensuremath{\frac{\partial}{\partial{#2}}{#1}}}

\newcommand*{\mdiff}[3]{\ensuremath{\frac{\d{^{#3}}}{\d{#2}^{#3}}{#1}}}
\newcommand*{\pmdiff}[3]{\ensuremath{\frac{\partial{^{#3}}}{\partial{#2}^{#3}}{#1}}}

\newcommand*{\at}[1]{\ensuremath{\biggr\rvert_{#1}}}
\renewcommand*{\i}{\ensuremath{\text{i}}}
\newcommand{\distlike}{\ensuremath{\sim}}
\newcommand{\normaldist}[2]{\ensuremath{\mathcal{N}\left(#1, #2\right)}}
\newcommand{\asnormaldist}[2]{\distlike\normaldist{#1}{#2}}
\renewcommand*{\vec}[1]{\ensuremath{{\bm{#1}}}}
\newcommand*{\mat}[1]{\vec{#1}}
\newcommand*{\transpose}[1]{{#1}^{\mkern-1.5mu\mathsf{T}}}
\newcommand*{\invert}[1]{{#1}^{-1}}
\newcommand*{\tvec}[1]{\transpose{\vec{#1}}}
\newcommand*{\tmat}[1]{\tvec{#1}}
\newcommand*{\rgramian}[1]{{#1}\transpose{#1}}
\newcommand*{\lgramian}[1]{\transpose{#1}{#1}}
\newcommand*{\indicator}[1]{\ensuremath{\mathbb{1}\left\{#1\right\}}}
\newcommand*{\invmat}[1]{\ensuremath{\invert{\mat{#1}}}}
\newcommand*{\iprod}[3]{\ensuremath{\ifstrempty{#1}{\transpose{#2} {#3}}{\transpose{#1} {#2} {#3}}}}
\newcommand*{\ciprod}[2]{\ensuremath{\left\langle{#1},{#2}\right\rangle}}
\newcommand*{\miprod}[2]{\iprod{#1}{#2}{#1}}
\newcommand*{\enm}[1]{\left\langle{#1}\right\rangle}
\newcommand*{\grad}{\mathrm{grad}}
\renewcommand*{\div}{\mathrm{div}}
\newcommand*{\bigO}[1]{\ensuremath{\mathcal{O}\left({#1}\right)}}
\begin{document}

\title{Report for the semester thesis ``Development of a Monte Carlo algorithm for optimal control problems''}

\author{\IEEEauthorblockN{Stefano Weidmann}
\IEEEauthorblockA{\textit{Institute of fluid dynamics} \\
\textit{ETH Zurich}\\
Zurich, Switzerland \\
\href{mailto:stefanow@student.ethz.ch}{stefanow@student.ethz.ch}}}
%\and
%\IEEEauthorblockN{2\textsuperscript{nd} Given Name Surname}
%\IEEEauthorblockA{\textit{dept. name of organization (of Aff.)} \\
%\textit{name of organization (of Aff.)}\\
%City, Country \\
%email address}
%\and
%\IEEEauthorblockN{3\textsuperscript{rd} Given Name Surname}
%\IEEEauthorblockA{\textit{dept. name of organization (of Aff.)} \\
%\textit{name of organization (of Aff.)}\\
%City, Country \\
%email address}
%\and
%\IEEEauthorblockN{4\textsuperscript{th} Given Name Surname}
%\IEEEauthorblockA{\textit{dept. name of organization (of Aff.)} \\
%\textit{name of organization (of Aff.)}\\
%City, Country \\
%email address}
%\and
%\IEEEauthorblockN{5\textsuperscript{th} Given Name Surname}
%\IEEEauthorblockA{\textit{dept. name of organization (of Aff.)} \\
%\textit{name of organization (of Aff.)}\\
%City, Country \\
%email address}
%\and
%\IEEEauthorblockN{6\textsuperscript{th} Given Name Surname}
%\IEEEauthorblockA{\textit{dept. name of organization (of Aff.)} \\
%\textit{name of organization (of Aff.)}\\
%City, Country \\
%email address}
%}

\maketitle

\begin{abstract}
This document is a model and instructions for \LaTeX.
This and the IEEEtran.cls file define the components of your paper [title, text, heads, etc.]. *CRITICAL: Do Not Use Symbols, Special Characters, Footnotes, 
or Math in Paper Title or Abstract.
\end{abstract}

\begin{IEEEkeywords}
a, b, c
\end{IEEEkeywords}

\section{Problem description}
We model a cross section of an oil field as a two dimensional square $\Omega := [0, 1]^2.$ In the oilfield, there are two phases: water and oil. 
At the lower left corner $(0, 0)$, we know the pressure $p(t)$. Opposite of that, at $(1, 1)$ a well is located.
There we can measure the pressure $p_\text{well}(t)$ as well as the volumetric outflow ${Q}_\text{well}(t)$ per unit area.

The flow rates for both phases are described by Darcy's law
\begin{equation}
\label{flowrateWater}
\vec{q}_\text{w} = -\frac{k k_\text{rel, w}}{\mu_\text{w}} \grad(p),
\end{equation}
for water and
\begin{equation}
\label{flowrateOil}
\vec{q}_\text{o} = -\frac{k k_\text{rel, o}}{\mu_\text{o}} \grad(p)
\end{equation}
for oil.
Here, $p(\vec{x}, t)$ is the pressure, $\mu_\text{o}$,$\mu_\text{w}$ are dynamic viscosities for oil and water, $k(\vec{x}, t)$ is the permeability.
$k_\text{rel, o}(S)$,$k_\text{rel, w}(S)$ are relative permeabilities and depend quadratically on the saturation of water $S \in [0, 1]$:
\begin{align}
k_\text{rel, o} &= (1 -S)^2\\
k_\text{rel, w} &= S^2.
\end{align}

$\vec{q}_\text{o}(\vec{x}, t)$, $\vec{q}_\text{o}(\vec{x}, t)$ finally are the volumetric flow rates per unit area.

The saturation $S$ is not assumed constant but instead is transported according to the equation
\begin{equation}
\label{stransp}
\pdiff{\phi S}{t} + \div(\vec{q}_\text{w}) = Q_\text{well}\delta(\vec{x} - \begin{pmatrix}1\\1\end{pmatrix})
\end{equation}

The saturation 
\begin{equation}
\vec{q}_\text{tot}(, t) \overset{!}{=} q_\text{well}(t).
\end{equation} 


\end{document}
